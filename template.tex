\documentclass[final]{beamer} % beamer 3.10: do NOT use option hyperref={pdfpagelabels=false}!

%\documentclass[final,hyperref={pdfpagelabels=false}]{beamer} % beamer 3.07: get rid of beamer warnings

\mode<presentation>{\usetheme{dsmanchester}}

% Portrait, A0, ?? scale =?
\usepackage[orientation=portrait, size=a0, scale=1.3]{beamerposter}

% Set height of page for use in arranging text boxes, after accounting for
% heading, footer etc.
\newlength{\columnheight}
\setlength{\columnheight}{105cm}

% Define the header information
\title{Title \\ \vspace{0.3em} second line of title}
\author{
  \textit{Main author}$^*$, Another one$^*$ and another author$^\dagger$}
\institute{$*$ A University, Somewhere \\
  $\dagger$ Somewhere else}


% Put any other packages or custom macros in here:



% Now start the actual poster
\begin{document}

% Header is automatically generated from the command defined in the theme
% file, the information above and the logos in the logo folder.


% Start of the body:
\begin{frame}
  \begin{columns}

    %% Left column:
    %% ============================================================
    \begin{column}{0.47\textwidth}

      % For some reason we need parbox to get it to arrange the boxes
      % with evenly spaced gaps
      \parbox[t][\columnheight]{\textwidth}{

        \vfill % vfill between every block so that everything is
               % automatically nicely spaced out.

        \begin{block}{Abstract}
          \begin{itshape}   % italic abstract
              \begin{itemize}
              \item Some key points
              \item I discovered something interesting!
              \end{itemize}
            \end{itshape}
        \end{block}

        \vfill

        \begin{block}{\boxnumber Main block number one}
        \end{block}

        \vfill

      } % end of parbox
    \end{column}

    %% Right column:
    %% ============================================================
    \begin{column}{0.47\textwidth}
      \parbox[t][\columnheight]{\textwidth}{

        \vfill

        \begin{block}{\boxnumber FEM results without magnetostatics}
          Just use any old latex that you like in here
        \end{block}

        \vfill

      } % end of parbox
    \end{column}

  \end{columns}
\end{frame}

\end{document}




%%% Local Variables:
%%% mode: latex
%%% TeX-master: t
%%% End:
