\documentclass[final]{beamer} % beamer 3.10: do NOT use option hyperref={pdfpagelabels=false}!

%\documentclass[final,hyperref={pdfpagelabels=false}]{beamer} % beamer 3.07: get rid of beamer warnings

\mode<presentation>{\usetheme{dsmanchester}}

% Increase spacing in lists
\newlength{\wideitemsep}
\setlength{\wideitemsep}{\itemsep}
\addtolength{\wideitemsep}{10pt}
\let\olditem\item
\renewcommand{\item}{\setlength{\itemsep}{\wideitemsep}\olditem}

\usepackage[T1]{fontenc}

%% Bibliograpy Tweaks
%% ============================================================

\usepackage[backend=biber,
  natbib=true,
  citestyle=numeric,
  firstinits=true, % Only print initials of first names
  isbn=false,
  doi=false,
]{biblatex}

%% Use tiny font size for citations
\renewcommand*{\bibfont}{\footnotesize}

\addbibresource{citations.bib}

% A superscript ciation command
\DeclareCiteCommand{\supercite}[\mkbibsuperscript]
  {\iffieldundef{prenote}
     {}
     {\BibliographyWarning{Ignoring prenote argument}}%
   \iffieldundef{postnote}
     {}
     {\BibliographyWarning{Ignoring postnote argument}}}
  {\usebibmacro{citeindex}%
   \bibopenbracket\usebibmacro{cite}\bibclosebracket}
  {\supercitedelim}
  {}

% Use superscript cite instead of normal ones
\let\cite=\supercite



% Portrait, A0, ?? scale =?
\usepackage[orientation=portrait, size=a0, scale=1.3]{beamerposter}

% Set height of page for use in arranging text boxes, after accounting for
% heading, footer etc.
\newlength{\columnheight}
\setlength{\columnheight}{105cm}

\newcommand{\email}{david.shepherd-2@man.ac.uk}

\newcommand{\uomaddr}{The University of Manchester, Manchester, M13 9PL, UK}

\title[sd_stiffness]{\Huge{Discretisation induced stiffness in \\ \vspace{0.3em} micromagnetic simulations}}
\author[David Shepherd]{
  \textit{David Shepherd}$^*$ (\texttt{\email{}}), Jim Miles$^*$, Matthias Heil$^\dagger$ and Milan Mihajlovi\'{c}$^*$}
\institute{$*$ NEST Group, School of Computer Science, \uomaddr{} \\
  $\dagger$ School of Mathematics, \uomaddr{}}



% New way to define commands that allows multiple subscripts
\makeatletter
\newcommand\newsubcommand[3]{\newcommand#1{#2\sc@sub{#3}}}
\def\sc@sub#1{\def\sc@thesub{#1}\@ifnextchar_{\sc@mergesubs}{_{\sc@thesub}}}
\def\sc@mergesubs_#1{_{\sc@thesub#1}}
\makeatother

% Some new maths commands:
\newcommand{\pd}[2]{\frac{\partial #1}{\partial #2}}
\newcommand{\mv}{\mathbf{m}}
\newcommand{\hv}{\mathbf{h}}
\newsubcommand{\happ}{\mathbf{h}}{{\text{ap}}} %applied
\newsubcommand{\hms}{\mathbf{h}}{\text{ms}} % magnetostatic/demag
\newsubcommand{\hex}{\mathbf{h}}{\text{ex}} % exchange
\newsubcommand{\hca}{\mathbf{h}}{\text{ca}} % crystalline ansiotropy
\newsubcommand{\hthm}{\mathbf{h}}{\text{th}} % thermal
\newcommand{\nH}{H_{\mathbb{n}}} % A "magnitude" of H for normalisation
\newcommand{\nF}{F_{\mathbb{n}}} % A "magnitude" of energy for normalisation


% ie and eg
\newcommand{\ie}{\textit{i.e.} }
\newcommand{\cf}{\textit{c.f.} }
\newcommand{\eg}{\textit{e.g.} }


\begin{document}
\begin{frame}

  \begin{columns}

    %% Left column:
    %% ============================================================
    \begin{column}{0.47\textwidth}

      % For some reason we need parbox to get it to arrange the boxes
      % with evenly spaced gaps
      \parbox[t][\columnheight]{\textwidth}{

        \begin{block}{Abstract}  % italic abstract
            \begin{itshape}
              \begin{itemize}
              \item Micromagnetic models are sometimes stiff (explicit time integration is slow) but when and why?
              \item Find that it can occur due to spatial discretisation alone.
              \item Find that FEM/BEM magnetostatic calculations increases stiffness.
              \item Find that decreasing damping parameter increases stiffness.
              \end{itemize}
            \end{itshape}
        \end{block}

        \vfill

        \begin{block}{\boxnumber The Landau-Lifshitz-Gilbert equation}
          \begin{itemize}
          \item Dynamic micromagnetic models centre around solving the
            Landau-Lifshitz-Gilbert equation (LLG).

          \item Partial differential equations for behaviour of magnetisation in a
            ferromagnetic material over time.

          \item The non-dimensional Landau-Lifshitz-Gilbert equation is:
            \begin{equation}
              \pd{\mv}{t} = \underbrace{-\mv \times \hv[\mv]}_{\text{precession}}
              + \underbrace{\alpha \mv \times \pd{\mv}{t}}_{\text{damping}}
            \end{equation}

            where the field terms are
            \begin{equation}
              \hv[\mv] = \underbrace{\happ}_{\text{applied }}
              + \underbrace{\hms[\mv]}_{\text{magnetostatic}}
              % + \underbrace{\hca[\mv]}_{\text{anisotropy}}
              + \underbrace{\nabla^2 \mv}_{\text{exchange}}.
            \end{equation}
          \end{itemize}
        \end{block}

        \vfill

        \begin{block}{\boxnumber Approaches to numerical solution of the LLG}
          \begin{itemize}
          \item Discretise LLG in space (\eg macrospin, finite difference, finite element).
          \item Calculate magnetostatic effects (\eg Fourier transform, FEM/BEM).
          \item Discretise in time (\eg Runge-Kutta, BDF2, implicit midpoint rule). Ideally
            sufficiently \emph{stable}: don't need high time refinement to stop the solution exploding.
          \end{itemize}
        \end{block}

        \vfill

        \begin{block}{\boxnumber Implicit vs explicit methods} 
          \begin{itemize} 
          \item Explicit methods (\eg most Runge-Kutta methods, Predictor-Corrector methods):
            \begin{itemize}
            \item Calculate values at next time step in terms of values at previous steps: \emph{each step is cheap}.
            \item Unstable for some problems: many extra steps needed for stability.
            \end{itemize}

          \item Implicit methods (\eg BDF2, implicit midpoint rule):
            \begin{itemize}
            \item Include value at the next step in the calculation: requires solution of a system of equations. 
            \item More expensive but \emph{always stable}.
            \end{itemize}

          \item ``Stiff problem''
            \begin{itemize}
            \item When explicit methods require so many additional steps that they become inefficient. 
            \item Can occur due to physics, \eg models of chemical reactions.
            \item Can also occur due to fine spatial discretisations.
            \end{itemize}
          \end{itemize}
        \end{block}

        \vfill

        \begin{block}{\boxnumber The experiment} 
          \begin{columns}
            \begin{column}{0.6\textwidth}
              \begin{itemize}
              \item Examine stiffness of a simple micromagnetic problem: coherent reversal of a small spherical nanoparticle\cite{Mallinson2000}, as shown in figure.
              \item Try with single macrospin and with various finite element meshes.
              \item Try with and without magnetostatics (using FEM/BEM method\cite{Knittel2011}).
              \item Try various time integration methods
                \begin{itemize}
                \item Explicit 2nd order Runge-Kutta (RK2).
                \item Implicit midpoint rule (IMR)\cite{DAquino2005}.
                \item Semi-implicit midpoint rule (SIMR): implicit LLG with explicit magnetostatics\cite{Shepherd2014}.
                \end{itemize}
              \item Examine maximum stable time steps for each while varying parameters.
              \end{itemize}
            \end{column}

            \begin{column}{0.4\textwidth}
              \begin{figure}[h]
                \centering
                \includegraphics[width=\textwidth]{images/sphere.pdf}
                \caption{Spherical nanoparticle of radius 1 exchange length in an applied field $\happ = [0, 0, -1.1]$. Initial magnetisation is slightly off the $z$-axis.}
                \label{fig:sphere}
              \end{figure} 
            \end{column}

          \end{columns}
         
      \end{block}


      \vfill

      \begin{block}{\boxnumber Macrospin results}
        \begin{itemize}
        \item Maximum time step tested is stable for all time integration methods.
        \item Same with/without magnetostatics and with $\alpha = 1.0, 0.1$ or $0.01$.
        \item Therefore problem is not physically stiff.
        \end{itemize}
      \end{block}

        \vfill


        % \begin{block}{\boxnumber An initial test case}
        %   \begin{columns}

        %     \begin{column}{0.65\textwidth}

        %       \begin{itemize}
        %       \item Small ellipsoidal nanoparticle (single domain, spatially constant magnetic field)
        %       \item Strong uniaxial anisotropy: not much happens at first then suddenly switches. So time adaptivity very helpful.
        %       \item Field applied along $-z$ axis.
        %       \end{itemize}
        %     \end{column}
        %     \begin{column}{0.3\textwidth}
        %       \includegraphics[width=0.9\textwidth]{images/ellipsoid.eps}
        %     \end{column}
        %   \end{columns}
        % \end{block} 
        % \vfill

      } % end of parbox
    \end{column}

    %% Set up main right column:
    %% ============================================================
    \begin{column}{0.47\textwidth}
      \parbox[t][\columnheight]{\textwidth}{

        \vfill

        \begin{block}{\boxnumber FEM results without magnetostatics}
          \begin{figure}[h]
            \begin{center}
            \includegraphics[width=\textwidth]{images/stability_disabled.pdf}
            \label{fig:stable-step-no-ms}
            \vspace{-1.5em} % Hack to fix caption spacing
            \caption{Maximum stable steps vs mesh fineness without magnetostatics. Dashed line shows RK2 time step needed for equal efficiency of IMR and RK2 method.}
          \end{center}

          \end{figure}
          \begin{itemize}
          \item Stiffer as spatial discretisation is improved.
          \item Stiffer as damping decreases.
          \end{itemize}
        \end{block}

        \vfill

        \begin{block}{\boxnumber FEM results with magnetostatics}
          \begin{figure}[h]
            \begin{center}
            \includegraphics[width=\textwidth]{images/stability_decoupled}
            \label{fig:stable-step-ms}
            \vspace{-1.5em} % Hack to fix caption spacing
            \caption{Maximum stable steps vs mesh fineness with FEM/BEM magnetostatics.}
          \end{center}
          \end{figure}
          \begin{itemize}
          \item Qualitatively the same as above, but more stiff.
          \item Semi-implicit method (implicit LLG, explicit magnetostatics) does not show any stability issues.
          \end{itemize}
        \end{block}

        % \begin{block}{\boxnumber New adaptive implicit midpoint rule} 
        %   \begin{itemize}

        %   \item Implicit midpoint rule is a well known geometric integrator
        %     \begin{equation}
        %       \label{eq:imr}
        %       \frac{\mv_{n+1} - \mv_n}{\Delta t} = 
        %       \frac{\mv_{n+1} + \mv_n}{2} \times \left( \hv\left[\frac{\mv_{n+1} + \mv_n}{2} \right]
        %         + \frac{\mv_{n+1} - \mv_n}{\Delta t} \right).
        %     \end{equation}

        %   \item Analysis shows conservation properties retained with adaptive time step.

        %   \item Use a third order explicit backwards Euler method to get an error estimate.

        %   \item Automatically select appropriate size for next step.

        %   \item Example below for spatially independent LLG, \ie single spin in a magnetic field.

        %   \end{itemize}

        %   \begin{center}
        %     \includegraphics[width=0.8\textwidth]{images/adaptivity_fig.eps}
        %   \end{center}

        %   \begin{center}
        %     \includegraphics[width=0.8\textwidth]{images/midpoint-conservation}
        %   \end{center}

        % \end{block}

        % \vfill

        % \begin{block}{\boxnumber Conclusions}
        %   \begin{itemize}
        %   \item Gives perfect accuracy of magnetisation length unlike standard adaptive methods.
        %   \item Gives $\sim 3$ orders of magnitude better accuracy in energy loss that standard adaptive methods.
        %   \item More user-friendly and efficient than standard midpoint method due to step size adaptivity.
        %   \end{itemize}
        % \end{block}

        % \vfill

        % \begin{block}{\boxnumber Extension to FEM/BEM model}
        %   \begin{itemize}
        %   \item Requires fully implicit/coupled method for energy conservation.
        %   \item Looking at efficient solvers for such a method (preconditioned Krylov solve, hierarchical BEM matrices).
        %     \texttt{oomph-lib}\cite{oomph-lib-website}.
        %   \end{itemize}
        % \end{block}


        \vfill

        \begin{block}{\boxnumber Conclusions}
          \begin{itemize}
          \item Stiffness in micromagnetic simulations can arise purely from a fine spatial discretisation.
          \item For realistic dynamic models (magnetostatics, $\alpha = 0.01$) the problem we examined is always stiff.
          \item The semi-implicit method tested here shows no stability issues.
          \end{itemize}
        \end{block}

        \vfill

        \begin{block}{\boxnumber Acknowledgements}
          This project is funded by the Engineering and Physical Sciences
          Research Council (EPSRC) through the NoWNano Doctoral Training Centre on
          grant number EP/G01705/1.
        \end{block}

        \vfill

        \begin{block}{\boxnumber References}
          \printbibliography
        \end{block}

        \vfill

      } % end of parbox
    \end{column}

  \end{columns}
\end{frame}

\end{document}




%%% Local Variables:
%%% mode: latex
%%% TeX-master: t
%%% End:
